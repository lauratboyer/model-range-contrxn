\documentclass{article}
\usepackage{fullpage}
\usepackage{graphicx}
\usepackage{amsmath}

\newcommand{\capnow}{}
\newcommand{\addcenterfig}[3][Caption to be completed]{
  \begin{figure}[!h]
    \begin{center}
      \includegraphics[width=\textwidth]{#3}
     \caption{#1 \label{#2}}
    \end{center}
  \end{figure}
}

\begin{document}

Must read Freckleton et al 2005 as their model is similar and they get
the right shape for the AOR (Also check out Freckleton 2006). Holt et al 2002 and 1997 developed
different models for AOR, to check too.

Time-lags: depending on colonization/extinction rates there could be a
lag between the initial decline in abundance and when it is felt --
see Conrad et al 2001 for an example with a butterfly. I could model
this. \\

Holt, A.R., Warren, P.H. \& Gaston, K.J. (2002) The importance of
biotic interactions in abundance-occupancy relationships. Journal of
Animal Ecology, 71, 846–854.

See also rescue effect to low density sites (Gonzalez 1998) in years
of low occupancy.
\section{Introduction}
Matthysen (2005) notes that even though there has been many
theoretical models on DD dispersal, empirical studies remain few. See
introduction for more, and citing papers.

Using proportion of local cell density is simplification but quite
realistic for schooling fish as they are unlikely to move on their own
to different destinations.

\subsection{Implementation of density-dependent dispersal in the
  literature}
Travis et al (1999) had an evolutionary individual-based
model. Lattice with periodic boundaries. In their model the only
instance where DDD did not evolve was when there was no cost to
dispersal. No equation but cost of dispersal implemented as
probability of failing to reach destination (after which the
individual presumably dies/disappears from lattice).

Some models include a cost to dispersal -- example failure to reach
destination. Dispersal is a linear function of natal population
density.

\section{Relevant literature}
Trakhtenbrot, A., Nathan, R., Perry, G. \& Richardson, D.M. (2005) The
importance of long-distance dispersal in biodiversity
conservation. Diversity and Distributions, 11, 173–181. \\

\section{Overview of model}

This is a spatially explicit model with cells set-up in a lattice
grid. Populations dynamics in each cell follow a logistic model, that
is, for each cell two main parameters are defined, the carrying
capacity $K$ and the \emph{per capita} growth rate $r$. Population at
the next time step is then a function of population at the previous
timestep and the expected population growth given how close abundance
is to the carrying capacity:
\begin{equation}
N_{t+1} = N_t + N_t \times r \times (1-\frac{N_t}{K})
\end{equation}

It is useful to break down the logistic part as it performs two roles:
adding individuals (growth, reproduction, $r_G$ thereafter) and removing individuals
(natural mortality, $r_M$ thereafter).
\begin{equation}
\displaystyle N_{t+1}=N_t + \underbrace{N_t \times
  r_M}_{\textstyle{growth}} -
\underbrace{N_t \times r_M \times-\displaystyle\frac{N_t}{K}}_{\textstyle{mortality}}
\end{equation}

In the geographic range of a species we would expect growth and
mortality rates to change, and not necessarily in synchrony. As a
simple example, for mobile animals reproduction (growth) might only
occur in certain parts of the range while mortality could occur
throughout the range.
Read to cite maybe: REVIEW: Identifying links between vital rates and
environment: a toolbox for the applied ecologist.

Dispersal distance, see: The evolution of dispersal distance in
spatially‐structured populations, Murrell et al 2002.
Dispersal between cells can be implemented in two ways. When it is
emigration-driven, individuals 'decide' to emigrate from cells based
on parameters pertaining to the cell itself (e.g. set emigration
rate at everytime step, or the cell is too crowded) and immigration to
cells is thus calculated passively from emigration in neighbour
cells. If it is immigration driven then individuals move to a target
cells based on features of that cell -- e.g. the environment is
optimal. In that case emigration becomes a function of how
'attractive' neighbour cells are, compared to the currently occupied
cell (see also gravity models).\\

For now in the model dispersal is implemented as emigration-driven. At
each time-step a proportion of the current cell abundance is designed
to emigrate \emph{after reproducing}. This proportion can either be
constant or it can be a function of how close current cell abundance
is to carrying capacity. This is done to produce greater emigration
rate as the cell becomes 'crowded'. Emigration is set at a constant
rate ($E_{base}$) when N is smaller than half of the carrying capacity, and
increases linearly from that level to a maximum emigration rate
($E_{max}$), which is reached at $K$.\\

See studies cited in intro by Matthysen to see how DD dispersal was
modelled.

Also: it could be argued that fish can use chemical signals in the
water as cues to inform where they should disperse -- e.g. see paper
by Zimmer and Butman 2000 Chemical signaling processes in the marine
environments. Mostly invertebrates studies but still might be
relevant. Also paper by Southwood and al 2008 on using environmental
cues to reduce bycatch in longline fisheries -- reviews cues used by
large pelagics.

\section{Model behaviour given parameter set}
\subsection{No dispersal}
When there is no dispersal between cells each cell reaches carrying
capacity following a smooth logistic curve (Figure \ref{nodisp}). This
also happens if $K$ varies between cells (Figure \ref{nodisp-core}).

\addcenterfig{nodisp}{Theo-mod_range-contrxn_emigbase-0_emigmax-0_habtype-even.pdf}
\addcenterfig{nodisp-core}{Theo-mod_range-contrxn_emigbase-0_emigmax-0_habtype-core.pdf}

Could change the K proportion at which dispersal starts occuring.
Could have an option that if it is a 'vital' cell, dispersal happens
only after threshold, if it is a non-vital cell, dispersal happens all
the time.
\subsection{Constant dispersal}


\subsection{Natal cell driven density-dependent dispersal}

\renewcommand{\capnow}{Under low, constant and even dispersal rates,
  core cells reach an equilibrium lower than their defined
  K, while edge cells reach an equilibrium higher than their defined K.}
\addcenterfig[\capnow]{even-low-disp}{Theo-mod_range-contrxn_emigbase-01_emigmax-01_habtype-core}

\subsection{Adding fishing}
When there is fishing and no dispersal.

When there is dispersal as fishing increases abundance in cells
becomes move even.

If there is fishing only in the core, abundance in core cells is
higher if there is dispersal.
If there is little dispersal the cell the furthest away from migrants
goes extinct first.

Under high core fishing mortality core cells will only subsist if
there is dispersal.

Also if there is even a low base emigration rate for all cells,
abundance is declines smoothly around the core ... otherwise sharp
square. Compare map.abund() with emig.base = 0 vs emig.base=0.1
\subsection{Fishing as first step}
\subsection{Adding core r}
When adding core r (i.e. r in edge cells = 0) edge cells reach higher
K because the logistic term acts makes it converge to K (when N is
greater than K the term becomes negative)

\subsection{shape of AOR vs assumption}
if no emigration before certain threshold is reach then flat then
increases, if emigration at first and then reach plateau, then other
type. right now modelling first kind... how can I get the third type?

\subsection{Discussion on ways to implement dispersal}
If dispersal is a function of local carrying capacity then there
should not be that much biomass distribution. If it is a function of
global/landscape carrying capacity, then there should be more
movement... but should also account for K gradient between cells
(i.e. individuals need to be rewarded for moving to cells with lower
N/K)... e.g. could be that at each time step proportion of migrants is
calculated, but instead of dispersing evenly between neighbours they
disperse preferentially to neighbours with lower N/K.

NEXT*** Create relationship to implement neighbour preference. Could be
standardized proportion of 1-N/K... to make it more/less even first
divide by self, gradient between dividing by self to dividing by 1 (if
value is 1 full neighbour preference, if value is N/K no neighbour
preference).

This would be realistic for animals that follow a gradient of good
conditions, vs animals that follow a migration route no matter what.

\section{Developing hypotheses to test from data}
E.g. if range contraction due to high local mortality in edges,
vs. DDHS, then spatial distribution is the same BUT rates of
emigration by cell should be different.

(Would be good to email Holdsworth about this)

\section*{Literature on density-dependent dispersal}
Matthysen, E. (2005). Density-dependent dispersal in birds and
mammals. Ecography, 28(3),
403–416. doi:10.1111/j.0906-7590.2005.04073.x \\

McCarthy (1999) Effect of competition on native dispersal distance.\\

Travis, J. M. J., Murrell, D. J., \& Dytham, C. (1999). The evolution
of density–dependent dispersal. Proceedings of the Royal Society
B-Biological Sciences, 266(1431), 1837–1842. \\

Travis and Lambin (2000). Dispersal functions and spatial models:
expanding our dispersal toolbox. \\

Dens dependent examples: birds and mammals, see Matthysen, reptile see
Lena (1998) (cited in Matthysen), insects see citations in
Matthysen.\\

Bowler, D. E., \& Benton, T. G. (2005). Causes and consequences of
animal dispersal strategies: relating individual behaviour to spatial
dynamics. Biological Reviews. \\

Kokko, H., \& López-Sepulcre, A. (2006). From individual dispersal to
species ranges: perspectives for a changing world. Science, 313(5788),
789–791. doi:10.1126/science.1128566\\

\end{document}
